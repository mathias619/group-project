% !TEX TS-program = pdflatex
% !TEX encoding = UTF-8 Unicode

% This is a simple template for a LaTeX document using the "article" class.
% See "book", "report", "letter" for other types of document.

\documentclass[11pt]{article} % use larger type; default would be 10pt

\usepackage[utf8]{inputenc} % set input encoding (not needed with XeLaTeX)

%%% Examples of Article customizations
% These packages are optional, depending whether you want the features they provide.
% See the LaTeX Companion or other references for full information.

%%% PAGE DIMENSIONS
\usepackage{geometry} % to change the page dimensions
\geometry{a4paper} % or letterpaper (US) or a5paper or....
% \geometry{margin=2in} % for example, change the margins to 2 inches all round
% \geometry{landscape} % set up the page for landscape
%   read geometry.pdf for detailed page layout information

\usepackage{graphicx} % support the \includegraphics command and options

% \usepackage[parfill]{parskip} % Activate to begin paragraphs with an empty line rather than an indent

%%% PACKAGES
\usepackage{booktabs} % for much better looking tables
\usepackage{array} % for better arrays (eg matrices) in maths
\usepackage{paralist} % very flexible & customisable lists (eg. enumerate/itemize, etc.)
\usepackage{verbatim} % adds environment for commenting out blocks of text & for better verbatim
\usepackage{subfig} % make it possible to include more than one captioned figure/table in a single float
% These packages are all incorporated in the memoir class to one degree or another...

%%% HEADERS & FOOTERS
\usepackage{fancyhdr} % This should be set AFTER setting up the page geometry
\pagestyle{fancy} % options: empty , plain , fancy
\renewcommand{\headrulewidth}{0pt} % customise the layout...
\lhead{}\chead{}\rhead{}
\lfoot{}\cfoot{\thepage}\rfoot{}

%%% SECTION TITLE APPEARANCE
\usepackage{sectsty}
\allsectionsfont{\sffamily\mdseries\upshape} % (See the fntguide.pdf for font help)
% (This matches ConTeXt defaults)

%%% ToC (table of contents) APPEARANCE
\usepackage[nottoc,notlof,notlot]{tocbibind} % Put the bibliography in the ToC
\usepackage[titles,subfigure]{tocloft} % Alter the style of the Table of Contents
\renewcommand{\cftsecfont}{\rmfamily\mdseries\upshape}
\renewcommand{\cftsecpagefont}{\rmfamily\mdseries\upshape} % No bold!

%%% END Article customizations

%%% The "real" document content comes below...

\title{COMPILATION OF A CSP LIKE LANGUAGE}
\author{MATHIAS SSERWADDA 215003597}
%\date{} % Activate to display a given date or no date (if empty),
         % otherwise the current date is printed 

\begin{document}
\maketitle

\section{First section}

1 Introduction

This compiler has been developed with the aim of building a system for easy CSP modeling and solving. By taking advantage of the year-over-year increase in performance of SMT solvers, we hope that such a system can serve as an alternative to other decision procedures in many applications. The compiler can also be used for easy SMT benchmark generation.

2 Background to The Problem

Over the last decade there have been important advances in logic based techniques and tools. Advances have been especially significant in the field of propositional satisfiability (SAT), to the point that nowadays modern SAT solvers
can tackle real-world problem instances with millions of variables. Hence, SAT
solvers have become a viable engine for solving combinatorial discrete problems. For instance, an application that compiles specifications written in
a declarative modeling language into SAT is shown to give promising results. For some applications of SAT technology on industrial problems. Interesting comparisons between SAT and Constraint Satisfaction Problem (CSP)
encodings and techniques can be found in therein.
SAT techniques have been adapted for more expressive logics. For instance,
in the case of Satisfiability Modulo Theories (SMT), the problem is to decide
the satisfiability of a formula with respect to a decidable background theory,
such as the theory of linear (integer or real) arithmetic, arrays, lists, etc., or
combinations of them, in first order logic with equality [14]. Input formulas
are often syntactically restricted, for example, to be quantifier-free, so that
the problem is still decidable. Hence, an SMT instance is a generalization of
a Boolean SAT instance in which some propositional variables have been re-
placed by predicates from the underlying theories, and can contain formulas
Adaptations of SAT techniques to the SMT framework have been described therein.
The main application area of SMT is hardware and software verification.
However, the available theories do not restrict the usage of SMT to verification
problems and, in fact, they allow to encode many problems outside the verification area in a very natural way. There are already promising results in the
direction of adapting SMT techniques for solving CSPs, even in the case of combinatorial optimization (see, e.g., [10] for an application of an SMT solver on
an optimization problem, being competitive with the best weighted CSP solver
with its best heuristic on that problem). Fundamental challenges on SMT for
Constraint Programming (CP) and Optimization are detailed in [11].
Since the beginning of CSP solving, its holy grail has been to obtain a declarative language that allows users to easily specify their problem and forget about
the techniques required to solve it. 

3 Problem statement

Satisfiability Modulo Theories (SMT) is the problem of deciding the satisfiability
of a first-order formula with respect to some background first-order theory. That
is, an SMT instance is a first-order formula where some function and predicate
symbols have interpretations, according to the background theories.
Examples of theories are Equality and Uninterrupted Functions, Linear Integer Arithmetic, Linear Real Arithmetic, their fragments Integer Difference Logic
and Real Difference Logic, Arrays (useful in modeling and verifying software
programs), Bit-Vectors (useful in modeling and verifying hardware designs), or
combinations of them (see [13] for details). Most SMT solvers are restricted to
decidable quantifier free fragments of their logics, but this success for many applications. 
There are two main approaches to solve SMT instances, namely, the eager
and the lazy approach. In the eager approach, the formula is translated into
an propositional formula. This allows the use of o_-the-shelf SAT
solvers, but has important drawbacks like, e.g., exponential memory blow-ups.
For this reason, most, if not all, state-of-the-art SMT solvers implement a lazy
approach, which does not involve a translation into SAT. One of these approaches
is DPLL(T) [12], which consists of a general DPLL(X) engine, very similar in
nature to a SAT solver, whose parameter X is instantiated with a specialized
solver for a given theory T, producing a DPLL(T) system. For instance, they can be used to express constraints on the time elapsed between pairs of events. Since the satisfiability of conjunctions of difference literals can be reduced to the absence of negative cycles infinite weighted graphs, it can be decided in O(n3) time by the Bellman-Ford algorithm. For this reason, many solvers give a special treatment to such kind of literals.


4 The Scope

The input language deals with formulas, global constraints and the If Then Else
constraint.
The first part of a comprehension list is the pattern, i.e., the expression
that we want to generate. Currently, patterns must be arithmetic expressions
(in this example, the elements of the bidimensional array m). The rest of the
comprehension list is formed by two distinct kinds of expressions, namely, the
generators (in the example, i in [1..3] and j in [1..3], that expand the
pattern) followed by the filters, that restrict these expansions (e.g., i<>j).

5 Methodology: Compilation

The compiler has been implemented in Haskell. The compilation process has
two steps: the first step only checks for syntactic compliance and some minor
semantic details, and generates an intermediate code. The second step is the
one in charge of semantic analysis and the final SMT-LIB code generation. This
code generation step distinguishes between expressions that must be evaluated
at compilation time (such as, for instance, the expressions in the condition of the
If-Then-Else statement), or translated into SMT-LIB expressions (for instance
a basic constraint).
The names of the variables are preserved from the input.

6 Conclusion 

We have presented Simply, a tool for easy CSP modeling and solving, whose
main novelty is the generation of SMT problem instances in the standard SMT-
LIB format as output. Our aim is to take advantage from the improvements
that take place from year to year in SMT technology and methods, in order
to solve CSPs. Our tool can also serve as a CSP benchmark generator for SMT
solvers comparison. However, much work is still to be done in the development of
Simply to make it competitive with other tools for CSP solving. We distinguish
among three aspects.

Your text goes here.

\subsection{A subsection}

More text.
7. References

Marco Cadoli, Giovambattista Ianni, Luigi Palopoli, Andrea Schaerf, and
Domenico Vasile. NP-SPEC: an executable specification language for solving all problems in NP. Computer Languages, 26(2{4):165{195, July 2000.


\end{document}
